\section{Ejercicio 1}
Dado un entero positivo n, determinar el valor de $\floor{\log(n)}$.

\subsection*{Solución}
\subsubsection*{Algoritmo}
\begin{algorithm}[H]
\caption{Calcular $\floor{\log(n)}$}
\begin{algorithmic}[1]
\REQUIRE Un entero positivo $n$
\ENSURE El valor de $\floor{\log(n)}$
\STATE $count \gets 0$
\IF{$n \leq 1$}
    \RETURN $0$
\ENDIF
\WHILE{$n > 1$}
    \STATE $n \gets n/2$
    \STATE $count \gets count + 1$
\ENDWHILE
\RETURN $count$
\end{algorithmic}
\end{algorithm}

\subsubsection*{Análisis de complejidad}
\begin{enumerate}
\item \textbf{Tiempo de ejecución}
  \begin{itemize}
    \item Observa que en cada iteración del ciclo \textbf{while} la variable $n$ se divide entre 2. Por tanto, el número de iteraciones que el bucle realizará hasta que $n \leq 1$ es aproximadamente $\log_2(n)$.
    \item Cada iteración ejecuta un conjunto constante de operaciones (comparación, división y asignaciones), que podemos considerar de costo constante, $O(1)$.
    \item Como consecuencia, el tiempo total de ejecución es $O(\log n)$.
  \end{itemize}

\item \textbf{Espacio en memoria}
  \begin{itemize}
    \item El algoritmo solo utiliza un número constante de variables adicionales ($count$ y la propia $n$), por lo que el uso de espacio es constante.
    \item Por lo tanto, la complejidad espacial es $O(1)$.
  \end{itemize}
\end{enumerate}

En resumen, la complejidad temporal del algoritmo es $O(\log n)$ y la complejidad espacial es $O(1)$. 