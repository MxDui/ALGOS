\section{Ejercicio 5}
Dado un arreglo A de n enteros, ¿existe un elemento de A tal que aparece en A al menos $n/2$ veces?

\subsection*{Solución}
\subsubsection*{Algoritmo}
\begin{algorithm}[H]
\caption{Encontrar elemento mayoritario (apariciones $\geq n/2$)}
\begin{algorithmic}[1]
\REQUIRE Un arreglo \( A \) de \( n \) enteros.
\ENSURE Un elemento que aparece al menos \( n/2 \) veces, o \(\text{null}\) si no existe.
\STATE \( candidate \gets \text{null} \)
\STATE \( count \gets 0 \)
\FOR{cada \( x \) en \( A \)}
    \IF{\( count = 0 \)}
        \STATE \( candidate \gets x \)
        \STATE \( count \gets 1 \)
    \ELSE
        \IF{\( candidate = x \)}
            \STATE \( count \gets count + 1 \)
        \ELSE
            \STATE \( count \gets count - 1 \)
        \ENDIF
    \ENDIF
\ENDFOR
\STATE \( occurrence \gets 0 \)
\FOR{cada \( x \) en \( A \)}
    \IF{\( x = candidate \)}
         \STATE \( occurrence \gets occurrence + 1 \)
    \ENDIF
\ENDFOR
\IF{\( occurrence \ge \frac{n}{2} \)}
    \RETURN \( candidate \)
\ELSE
    \RETURN \(\text{null}\) \COMMENT{No existe elemento mayoritario}
\ENDIF
\end{algorithmic}
\end{algorithm}

\subsubsection*{Análisis de complejidad}
\paragraph{Complejidad Temporal:}
El algoritmo realiza dos recorridos sobre el arreglo \( A \):
\begin{itemize}
    \item El primer recorrido (líneas 3 a 13) para identificar un candidato mayoritario, con una complejidad de \( O(n) \).
    \item El segundo recorrido (líneas 14 a 18) para verificar que el candidato realmente aparece al menos \( n/2 \) veces, también con una complejidad de \( O(n) \).
\end{itemize}
Por lo tanto, la complejidad temporal total es:
\[
O(n) + O(n) = O(n)
\]

\paragraph{Complejidad Espacial:}
El algoritmo utiliza únicamente un número constante de variables adicionales (\( candidate \), \( count \) y \( occurrence \)). Por ello, la complejidad espacial es:
\[
O(1)
\]