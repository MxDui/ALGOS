\section{Ejercicio 5}
Dado un arreglo A de n enteros, ¿existe un elemento de A tal que aparece en A al menos $n/2$ veces?

\subsection*{Solución}
\subsubsection*{Algoritmo}
\begin{algorithm}[H]
\caption{Encontrar elemento mayoritario (apariciones $\geq n/2$)}
\begin{algorithmic}[1]
\REQUIRE Un arreglo \( A \) de \( n \) enteros.
\ENSURE Un elemento que aparece al menos \( n/2 \) veces, o \(\text{null}\) si no existe.
\STATE \( candidate \gets \text{null} \)
\STATE \( count \gets 0 \)
\FOR{cada \( x \) en \( A \)}
    \IF{\( count = 0 \)}
        \STATE \( candidate \gets x \)
        \STATE \( count \gets 1 \)
    \ELSE
        \IF{\( candidate = x \)}
            \STATE \( count \gets count + 1 \)
        \ELSE
            \STATE \( count \gets count - 1 \)
        \ENDIF
    \ENDIF
\ENDFOR
\STATE \( occurrence \gets 0 \)
\FOR{cada \( x \) en \( A \)}
    \IF{\( x = candidate \)}
         \STATE \( occurrence \gets occurrence + 1 \)
    \ENDIF
\ENDFOR
\IF{\( occurrence \ge \frac{n}{2} \)}
    \RETURN \( candidate \)
\ELSE
    \RETURN \(\text{null}\) \COMMENT{No existe elemento mayoritario}
\ENDIF
\end{algorithmic}
\end{algorithm}

\subsubsection*{Análisis de complejidad}
Para este algoritmo tenemos las siguientes operaciones:
\begin{table}[H]
    \centering
    \begin{tabular}{|c|c|c|}
        \hline
        \textbf{Código} & \textbf{Operaciones} & \textbf{Total} \\
        \hline
        $candidate \gets \text{null}$ & 1 asignación & 1 \\
        \hline
        $count \gets 0$ & 1 asignación & 1 \\
        \hline
        \textbf{for} (cada $x$ en $A$) & 1 salto, 2 lecturas ($A$) & 2 \\
        \hline
        if $count = 0$ & 1 salto, 1 lectura ($count$) y 1 comparación & 3 \\
        \hline
        $candidate \gets x$ & 1 escritura ($candidate$) & 1 \\
        \hline
        $count \gets 1$ & 1 escritura ($count$) & 1 \\
        \hline
        else if $candidate = x$ & 1 salto, 2 lecturas ($candidate$, $x$), 1 comparación & 4 \\
        \hline
        $count \gets count + 1$ & 1 lectura ($count$), 1 suma, 1 escritura & 3 \\
        \hline
        else & 1 salto & 1 \\
        \hline
        $count \gets count - 1$ & 1 lectura ($count$), 1 resta, 1 escritura & 3 \\
        \hline
        end if & 1 salto & 1 \\
        \hline
        end for & 1 salto & 1 \\
        \hline
        $occurrence \gets 0$ & 1 asignación & 1 \\
        \hline
        \textbf{for} (cada $x$ en $A$) & 1 salto, 2 lecturas ($A$) & 3 \\
        \hline
        if $x = candidate$ & 1 salto, 2 lecturas ($x$, $candidate$), 1 comparación & 4 \\
        \hline
        $occurrence \gets occurrence + 1$ & 1 lectura ($occurrence$), 1 suma, 1 escritura & 3 \\
        \hline
        end if & 1 salto & 1 \\
        \hline
        end for & 1 salto & 1 \\
        \hline
        if $occurrence \geq \frac{n}{2}$ & 1 salto, 2 lecturas ($occurrence$, n), 1 división & 4 \\
        \hline
        return $candidate$ & 1 salto, 1 lectura & 2 \\
        \hline
        else return $\text{null}$ & 1 salto & 1 \\
        \hline
        end if & 1 salto & 1 \\
        \hline
    \end{tabular}
    \caption{Análisis de complejidad del algoritmo de Boyer-Moore para encontrar el elemento mayoritario}
    \label{tabla:boyer-moore}
\end{table}
Donde las operaciones que se hacen siempre son del 1 al 4, de la 14 a la 16, la 21, uno de los dos return y la 25, por lo que al contarlas todas tenemos 1+1+2+3+3+4+3+4+1+1 = 23.
Mientras que para los ciclos for ocurre lo siguiente:


\paragraph{Complejidad Temporal:}
El algoritmo realiza dos recorridos sobre el arreglo \( A \):
\begin{itemize}
    \item El primer recorrido (líneas 3 a 13) para identificar un candidato mayoritario, con una complejidad de \( O(n) \).
    \item El segundo recorrido (líneas 14 a 18) para verificar que el candidato realmente aparece al menos \( n/2 \) veces, también con una complejidad de \( O(n) \).
\end{itemize}
Por lo tanto, la complejidad temporal total es:
\[
O(n) + O(n) + 23 = O(n)
\]

\paragraph{Complejidad Espacial:}
El algoritmo utiliza únicamente un número constante de variables adicionales (\( candidate \), \( count \) y \( occurrence \)). Por ello, la complejidad espacial es:
\[
O(1)
\]
