\section{Ejercicio 2}
Dado un arreglo A de n enteros y un entero objetivo K, ¿existen un par de índices $i \neq j$, tales que $A[i] + A[j] = K$?

\subsection*{Solución}
\subsubsection*{Algoritmo}
\begin{algorithm}[H]
\caption{Encontrar par de números que suman K}
\begin{algorithmic}[1]
\REQUIRE Un arreglo $A$ de $n$ enteros y un entero objetivo $K$
\ENSURE Verdadero si existe un par de índices $i \neq j$ tales que $A[i] + A[j] = K$
\STATE $hashMap \gets \{\}$
\FOR{$i \gets 0$ \TO $n-1$}
    \IF{$hashMap.containsKey(K - A[i])$}
        \RETURN true
    \ENDIF
    \STATE $hashMap.put(A[i], i)$
\ENDFOR
\RETURN false
\end{algorithmic}
\end{algorithm}

\subsubsection*{Análisis de complejidad}
\begin{tabular}{|c|c|c|}
  \hline
  Código & Operaciones & Total \\
  \hline
 $hashMap \gets \{\}$    & 1 asignación.   & 1 \\
  \hline
  for($i \gets 0$ to $n-1$)    & 1 salto (for), 1 escritura (i), 1 lectura (n), 1 resta & 4    \\
  \hline
  if $hashMap.containsKey(K - A[i])$ & 1 salto, 3 lecturas (A, i, K), 1 resta y una operación hash& 5 \\
  \hline
  Return True & 1 salto & 1 \\
  \hline
  end if & 1 salto & 1 \\
  \hline
  $hashMap.put(A[i], i)$ & 2 lecturas (A, i) y la operación del hash & 3 \\
  \hline
  end for & 1 salto & 1 \\
  \hline
  return false & 1 salto & 1
\end{tabular}
Las operaciones hash según la \href{https://docs.oracle.com/javase/8/docs/api/java/util/HashMap.html}{documentación de Java} son constantes.
Las que se hacen si o si son la asignación del Hash, el for y el return false, es decir 1+4+1 = 6, mientras que en una sola iteración hace 17 operaciones, ahora para el for ocurre lo siguiente:

\paragraph{Complejidad Temporal:}
El algoritmo realiza las siguientes operaciones:
\begin{itemize}
    \item Recorre el arreglo una sola vez, visitando sus $n$ elementos
    \item Para cada elemento realiza operaciones de hash (búsqueda e inserción) que son $O(1)$ en promedio
\end{itemize}
Por lo tanto, la complejidad temporal total es:
\[
O(17n+6)
\]
Pero nuevamente, se pueden quitar las constantes por lo que nos queda solamente $O(n)$

\paragraph{Complejidad Espacial:}
El algoritmo utiliza un \texttt{hashMap} que puede almacenar hasta $n$ elementos. Por ello, la complejidad espacial es:
\[
O(n)
\]

\paragraph{Resumen:}  
- **Tiempo Promedio:** \(O(n)\)  
- **Tiempo Peor Caso:** \(O(n^2)\) (escenario teórico con colisiones excesivas)  
- **Espacio:** \(O(n)\)

En conclusión, el algoritmo es eficiente en la mayoría de los casos prácticos, logrando encontrar el par de números que suman \(K\) en tiempo lineal con respecto al tamaño del arreglo.
