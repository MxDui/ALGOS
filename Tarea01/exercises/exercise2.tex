\section{Ejercicio 2}
Dado un arreglo A de n enteros y un entero objetivo K, ¿existen un par de índices $i \neq j$, tales que $A[i] + A[j] = K$?

\subsection*{Solución}
\subsubsection*{Algoritmo}
\begin{algorithm}[H]
\caption{Encontrar par de números que suman K}
\begin{algorithmic}[1]
\REQUIRE Un arreglo $A$ de $n$ enteros y un entero objetivo $K$
\ENSURE Verdadero si existe un par de índices $i \neq j$ tales que $A[i] + A[j] = K$
\STATE $hashMap \gets \{\}$
\FOR{$i \gets 0$ \TO $n-1$}
    \IF{$hashMap.containsKey(K - A[i])$}
        \RETURN true
    \ENDIF
    \STATE $hashMap.put(A[i], i)$
\ENDFOR
\RETURN false
\end{algorithmic}
\end{algorithm}

\subsubsection*{Análisis de complejidad}
\paragraph{Complejidad Temporal:}
El algoritmo realiza las siguientes operaciones:
\begin{itemize}
    \item Recorre el arreglo una sola vez, visitando sus $n$ elementos
    \item Para cada elemento realiza operaciones de hash (búsqueda e inserción) que son $O(1)$ en promedio
\end{itemize}
Por lo tanto, la complejidad temporal total es:
\[
O(n)
\]

\paragraph{Complejidad Espacial:}
El algoritmo utiliza un \texttt{hashMap} que puede almacenar hasta $n$ elementos. Por ello, la complejidad espacial es:
\[
O(n)
\]

\paragraph{Resumen:}  
- **Tiempo Promedio:** \(O(n)\)  
- **Tiempo Peor Caso:** \(O(n^2)\) (escenario teórico con colisiones excesivas)  
- **Espacio:** \(O(n)\)

En conclusión, el algoritmo es eficiente en la mayoría de los casos prácticos, logrando encontrar el par de números que suman \(K\) en tiempo lineal con respecto al tamaño del arreglo.
