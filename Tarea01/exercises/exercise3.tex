\section{Ejercicio 3}
Dado un entero positivo n, determinar la cantidad de números primos menores o iguales a n.

\subsection*{Solución}
\subsubsection*{Algoritmo}
\begin{algorithm}[H]
\caption{Contar números primos usando la Criba de Eratóstenes}
\begin{algorithmic}[1]
\REQUIRE Un entero positivo $n$
\ENSURE La cantidad de números primos menores o iguales a $n$
\STATE $esPrimo \gets [true] * (n+1)$ \COMMENT{Arreglo booleano inicializado en verdadero}
\STATE $esPrimo[0] \gets false$
\STATE $esPrimo[1] \gets false$
\STATE $contador \gets 0$
\FOR{$i \gets 2$ \TO $\sqrt{n}$}
    \IF{$esPrimo[i]$}
        \FOR{$j \gets i^2$ \TO $n$ \textbf{step} $i$}
            \STATE $esPrimo[j] \gets false$
        \ENDFOR
    \ENDIF
\ENDFOR
\FOR{$i \gets 2$ \TO $n$}
    \IF{$esPrimo[i]$}
        \STATE $contador \gets contador + 1$
    \ENDIF
\ENDFOR
\RETURN $contador$
\end{algorithmic}
\end{algorithm}

\subsubsection*{Análisis de complejidad}
Tenemos la siguiente tabla
\begin{table}[H]
    \centering
    \begin{tabular}{|c|c|c|}
        \hline
        \textbf{Código} & \textbf{Operaciones} & \textbf{Total} \\
        \hline
        $esPrimo \gets [true] * (n+1)$ & 1 asignación + 1 multiplicacion + 1 lectura (n) + 1 suma & 4 \\
        \hline
        $esPrimo[0] \gets false$ & 1 escritura & 1 \\
        \hline
        $esPrimo[1] \gets false$ & 1 escritura & 1 \\
        \hline
        $contador \gets 0$ & 1 escritura & 1 \\
        \hline
        \textbf{for} ($i \gets 2$ to $\sqrt{n}$) & 1 salto, 1 escritura ($i$), 1 lectura ($n$), 1 raíz cuadrada & 4 \\
        \hline
        if $esPrimo[i]$ & 1 salto, 1 lectura ($esPrimo[i]$) & 2\\
        \hline
        \textbf{for} ($j \gets i^2$ to $n$, step $i$) & 1 escritura, 1 multiplicación (i * i), 2 lectura & 4 \\
        \hline
        $esPrimo[j] \gets false$ & 1 escritura & 1 \\
        \hline
        end if & 1 salto & 1 \\
        \hline
        end for & 1 salto & 1 \\
        \hline
        \textbf{for} ($i \gets 2$ to $n$) & 1 salto, 1 escritura ($i$), 1 lectura ($n$) & 3 \\
        \hline
        if $esPrimo[i]$ & 1 salto, 2 lectura ($esPrimo[i]$, i) & 3 \\
        \hline
        $contador \gets contador + 1$ & 1 lectura ($contador$), 1 suma, 1 escritura & 3 \\
        \hline
        end if & 1 salto & 1 \\
        \hline
        end for & 1 salto & 1 \\
        \hline
        return $contador$ & 1 salto, 1 lectura & 2 \\
        \hline
    \end{tabular}
    \caption{Análisis de complejidad del algoritmo de la Criba de Eratóstenes}
    \label{tabla:criba}
\end{table}

\paragraph{Complejidad Temporal:}
El algoritmo si o si hace las primeras 6 lineas, y las últimas, de la 11 a la 17, por lo que el algotimo tiene 13+14 = 27, en el ciclo for ocurre lo siguiente:

El algoritmo realiza las siguientes operaciones:
\begin{itemize}
    \item Inicialización del arreglo: $O(n)$
    \item Marcado de múltiplos: $O(n \log \log n)$
    \item Conteo final de primos: $O(n)$
\end{itemize}
Por lo tanto, la complejidad temporal total es:
\[
O(n \log \log n + 26)
\]
Pero como se pueden quitar las constantes solo nos queda $O(n \log \log n)$

\paragraph{Complejidad Espacial:}
El algoritmo utiliza un arreglo booleano de tamaño $n+1$. Por ello, la complejidad espacial es:
\[
O(n)
\] 
