\section{Ejercicio 3}
Dado un entero positivo n, determinar la cantidad de números primos menores o iguales a n.

\subsection*{Solución}
\subsubsection*{Algoritmo}
\begin{algorithm}[H]
\caption{Contar números primos usando la Criba de Eratóstenes}
\begin{algorithmic}[1]
\REQUIRE Un entero positivo $n$
\ENSURE La cantidad de números primos menores o iguales a $n$
\STATE $esPrimo \gets [true] * (n+1)$ \COMMENT{Arreglo booleano inicializado en verdadero}
\STATE $esPrimo[0] \gets false$
\STATE $esPrimo[1] \gets false$
\STATE $contador \gets 0$
\FOR{$i \gets 2$ \TO $\sqrt{n}$}
    \IF{$esPrimo[i]$}
        \FOR{$j \gets i^2$ \TO $n$ \textbf{step} $i$}
            \STATE $esPrimo[j] \gets false$
        \ENDFOR
    \ENDIF
\ENDFOR
\FOR{$i \gets 2$ \TO $n$}
    \IF{$esPrimo[i]$}
        \STATE $contador \gets contador + 1$
    \ENDIF
\ENDFOR
\RETURN $contador$
\end{algorithmic}
\end{algorithm}

\subsubsection*{Análisis de complejidad}
\paragraph{Complejidad Temporal:}
El algoritmo realiza las siguientes operaciones:
\begin{itemize}
    \item Inicialización del arreglo: $O(n)$
    \item Marcado de múltiplos: $O(n \log \log n)$
    \item Conteo final de primos: $O(n)$
\end{itemize}
Por lo tanto, la complejidad temporal total es:
\[
O(n \log \log n)
\]

\paragraph{Complejidad Espacial:}
El algoritmo utiliza un arreglo booleano de tamaño $n+1$. Por ello, la complejidad espacial es:
\[
O(n)
\] 